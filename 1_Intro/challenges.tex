\subsection{Challenges of integrtaing UTM with ATM}

Integrating \gls{UTM} into traditional \gls{ATM}is complex but crucial for ensuring safe and efficient airspace operations, primary due to the differing nature of \gls{UAS} and manned aircrafts. 
While \gls{UTM}, designed to manage drone missions, , it must be integrated seamlessly into the existing \gls{ATM} infrastructure to prevent accidents and enhance scalability. 
Both systems must work together, as unmanned flight systems need to detect and respond to other aircraft in emergencies, and vice versa \cite{flynex2020utm_atm}.

The growing complexity of \gls{ASM}, driven by the driven by the rapid expansion of commercial aviation, \gls{UAM}, and \glspl{UAV}, has led to increased air traffic volume.
As air traffic rises, and with the limited capacity of \glspl{ATCO} highlight the need for AI-based solutions, such as real-time data processing and predictive analytics, to improve system performance \cite{Ramachandran_2025}.
Despite existing automation, current systems often rely on rigid frameworks that lack the flexibility needed for dynamic environments  \cite{Meier_2024}.

Additionally, \gls{UAS} have unique performance characteristics that complicate their integration into the air traffic flow, often resulting in suboptimal use of airspace capacity.
\Glspl{UAV} typically operate across both controlled and uncontrolled airspaces, and since \glspl{ATCO} only manage controlled spaces, the lack of oversight in uncontrolled airspace raises the risk of collisions or accidents. 
As a result, \gls{UTM} systems are essential for ensuring safe and efficient \gls{UAV} operations across all airspaces \cite{Zsolt_2017}, highlighting the urgent need for scalable, flexible, and automated solutions in air traffic management.