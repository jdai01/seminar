\section{Introduction}

The rapid advancement of aerospace technology and urban infrastructure is driving the emergence of \gls{UAM} and autonomous airliners, reshaping the future of air transportation.
\Gls{UAM} refers to the use of \gls{eVTOL} aircraft to provide efficient, low-emission air travel within and around cities, aiming to alleviate ground traffic congestion and reduce travel times \cite{easa_uam}.
Simultaneously, the development of autonomous airliners (capable of operating with minimal or no human intervention) is gaining momentum, promising increased safety, operational efficiency, and cost-effectiveness in commercial aviation \cite{Vance_2019}.
Together, these innovations mark a significant shift toward smarter, more sustainable air transportation systems, supported by breakthroughs in \gls{AI}, sensor technology, and regulatory evolution. 

As the skies grow increasingly crowded with traditional aircraft, drones, and emerging \gls{eVTOL} vehicles, modernising \gls{ATM} systems becomes essential.
Traditional \gls{ATM} frameworks, designed for conventional aviation, are not equipped to handle the complexity and volume introduced by \gls{UAM} and autonomous operations \cite{Schuchardt_2023}.
To address this, the development of \gls{UTM} has emerged as a complementary solution, enabling the safe, scalable, and efficient integration of low-altitude, autonomous air traffic into national airspace systems \cite{Singh_2024}.
\Gls{UTM} leverages digital communication, real-time data sharing, and dynamic airspace access.
Together, \gls{ATM} and \gls{UTM} form the backbone of future-proofed aerial ecosystem, ensuring safety, reliability, and coordination across all types of airborne vehicles.

% Despite the promising potential of \gls{UTM}, integrating it seamlessly into legacy \gls{ATM} systems presents significant challenges.
% One of the primary issues is the technological and procedural gap between legacy air traffic infrastructure and the digital, decentralised nature of \gls{UTM} frameworks.
% Ensuring interoperability, real-time data exchange, and cohesive traffic coordination between manned and unmanned aircraft requires robust standards, secure communication protocols, and regulatory harmonisation.
% Compounding these challenges is the ongoing global shortage of \glspl{ATCO}, which places additional strain on already overburdened \gls{ATM} operations.
% This staffing deficit not only affects the efficiency of current airspace management but also limits the capacity to oversee and adapt to the influx of new air traffic types.
% Briding the gap between \gls{UTM} and \gls{ATM} will require collaborative efforts across industry stakeholders, regulaatory bodies, and technological innovators to build a resilient, scallable system capable of supporting the future of autonomous and \gls{UAM}.

Integrating \gls{UTM} into existing \gls{ATM} infrastructure presents a range of technical, operational, and organisational challenges \cite{Singh_2024}.
Traditional \gls{ATM} systems are already under pressure, with many countries facing a critical shortage of \glspl{ATCO}: an issue that hampers the capacity to manage even current levels of air traffic safely and efficiently \cite{eurocontrol2024digitalisation}.
Adding to this challenge is the rapid growth of \gls{UAS} and \gls{UAM} operations, which introduces unpredictable flight patterns, higher traffic density in low-altitude airspace, and the need for real-time, automated coordination \cite{Ramachandran_2025}.
These factors demand that \gls{UTM} systems be not only interoperable with legacy \gls{ATM} systems, but also highly robust, adaptive, and capable of autonomous decision-making.
Ensuring seamless integration while maintaining safety, reliability, and trust across both manned and unmanned aviation domains remains a core hurdle in realising the potential of next-generation air mobility.
