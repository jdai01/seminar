\section{Challenges of AI in ATM}

Despite the significant progress in enhancing \gls{ATM} systems through the integration of \gls{AI}, numerous challenges remain that hinder its seamless implementation. 
Malakis et al.~\cite{Malakis_2022} provide a comprehensive overview of these challenges, highlighting the complexities involved in technical, operational, organizational, and regulatory domains. 
% These challenges are multifaceted and span across various stakeholders and system levels.
% In this section, the key categories of challenges are presented and discussed under five main areas: Political/Regulatory, ANSP/Business, Technical, Operational, and Human Factors related to \glspl{ATCO}.


\subsection{Political / Regulatory}

The political and regulatory challenges in integrating \gls{AI} into \gls{ATM} include the persistent fragmentation of the European \gls{ATM} sector and state sovereignty issues, which complicate coordinated reform. 
The certification processes for \gls{AI}-related technologies remain complex due to their novelty, while outdated development assurance frameworks require adaptation, exemplified by the emergence of the Learning Assurance Concept. 
Legal ambiguity persists regarding the roles of Air Traffic Services Providers versus \gls{ATM} Data Service Providers. 
Moreover, harmonizing ethics policies, ensuring geographical redundancy of data centers, resolving cross-border risk-sharing agreements, and maintaining public trust through transparency and social acceptability pose additional hurdles.

\subsection{ANSP / Business}

From the perspective of Air Navigation Service Providers (ANSPs) and business stakeholders, challenges revolve around managing organizational transformation, legal and insurance uncertainties, and the risks associated with early adoption of emerging technologies. 
The unclear cost-benefit profile of \gls{AI} solutions, combined with complexities in defining service scopes and boundaries, further complicates investment decisions. 
There are also high costs for upgraded cybersecurity, and the transition disrupts established knowledge-sharing and training practices.
Effective change management strategies, including robust simulation and feedback loops involving operators, are critical to facilitating this technological shift.

\subsection{Technical}

Technically, the implementation of \gls{AI} in \gls{ATM} systems is impeded by issues such as the difficulty of sharing \gls{AI} infrastructure across national boundaries and the inherent dependence of machine learning models on high-quality, representative datasets. 
Challenges like the curse of dimensionality, lack of generalizability across operational contexts, and the need for tailored \gls{AI} solutions hinder scalability. 
Additionally, ensuring data integrity, developing resilient backup systems, and designing \gls{AI} capable of interpreting weak signals for decision-making are essential to maintaining safety and operational reliability.

\subsection{Operational}

Operational challenges include managing the increased complexity and interdependencies introduced by \gls{AI}, and synchronizing procedures between Air Traffic Services Units and the Network Manager. 
Current Concepts of Operations (CONOPS) often fail to accommodate \gls{AI} capabilities, necessitating new operational models. 
Issues with the explainability of AI outputs and poorly defined function allocation can leave controllers vulnerable when automation fails. 
A cohesive strategy is needed to harmonize AI integration across all stakeholders, including controllers, pilots, and airport operators, while mitigating the disruption to established communication and authority structures.

\subsection{ATCOs}

For \glspl{ATCO}, \gls{AI} integration presents challenges in maintaining situational awareness and adapting to changing roles. 
It disrupts long-standing coordination patterns among \glspl{ATCO} and between \glspl{ATCO} and flight crews, potentially undermining resilience. 
Increasing automation introduces unexpected behaviors and error types, requiring new cognitive skills and mental models to understand and manage \gls{AI} systems. 
Acceptance of new responsibilities, resistance due to job security concerns, de-skilling, attention management difficulties, and social impacts such as mobility and relocation further complicate human factors integration into \gls{AI}-enhanced \gls{ATM} environments.