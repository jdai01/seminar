\section{Conclusion and Outlook}

The integration of \gls{AI} into \gls{ATM} is transforming aviation through innovations like linear optimisation for flight planning, dynamic airspace sectorisation, and digital \gls{ATCO} assistants. 
These technologies aim to enhance operational efficiency, optimise airspace use, and reduce controller workload by supporting decision-making and automating routine tasks.

Despite this progress, significant challenges remain. 
Regulatory fragmentation, complex certification processes, data privacy concerns, and the need for explainable \gls{AI} all hinder seamless implementation. 
Operational models must adapt to \gls{AI} capabilities, while human factors such as \gls{ATCO} role changes, training, and trust in automation require careful consideration to ensure safety and acceptance.

Looking forward, the concept of fully autonomous, unmanned \gls{ATM} systems remains a distant but plausible goal. 
Ongoing research explores \gls{AI}-driven airspace management and autonomous coordination of \glspl{UAV}. 
Although technical, ethical, and regulatory hurdles persist, the potential for such systems highlights the importance of sustained innovation, collaboration, and a measured approach to future \gls{AI} adoption in \gls{ATM}.