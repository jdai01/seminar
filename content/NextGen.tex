\paragraph{Case Study:} \gls{NextGen} \cite{skybrary_nextgen}

\gls{NextGen} superceeds the \gls{NAS} of the United States, representing the evolution from a ground-based system of \gls{ATC} to satalite-based system of \gls{ATM}.
The combination of satallite-based and digital technologies and new procedures make air travel more convenient, predictable and environmental friendly in the US's increasing congested airspace.

It consists of five elements:
\begin{enumerate}
    \item \gls{ADS-B}: uses \gls{GPS} satellite signals to provide \glspl{ATCO} and pilots with much more accurate information that will keep aircraft safely separated in the sky and on runways.
    \item \gls{NDC}: using data communications to provide additional means of two-way communication, on top of the current voice communications between aircrew, \gls{ATC}, and \glspl{ATCO}.
    \item \gls{NNEW}: a single national weather information system that is updated in real time to provide a common weather picture across \gls{NAS}, and enable better air transportation decision making
    \item \gls{SWIM}: a single infrastructure and information management system to deliver data to many users, while also reducing the number and types of interfaces and systems, and reducing data redundancy
    \item \gls{NVS}: a single air/ground and ground/ground voice communications systemm, superceeding the current 17 different voice switching systems in the \gls{NAS}
\end{enumerate}

Through these elements, \gls{NextGen} assists with trajectory based operations of aircrafts and reducing any weather impacts in real time, providing pilots and \glspl{ATCO} with the same precise information transmitted via data communications, improving decision-making.
It also helps with the development of new procedures to improve the management of airports and terminals, increase the throughput to fly aircrafts in and out of many airports, especially high density airports.
