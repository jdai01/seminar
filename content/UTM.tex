\section{UAS Traffic Management (UTM)}

The goal of \gls{UTM} is to create a system that can integrate drones safely and efficiently into air traffic that is already flying in low-altitude airspaces. 
That way, package delivery and fun flights won't interfere with helicopters, airplanes, nearby airports or even safety drones being flown by first responders helping to save lives.
\cite{nasa_utm_2021}


\subsection{Differences compared to ATM}

The system is a bit different than the air traffic control system used by the \gls{FAA} for today's commercial airplanes.
UTM is based on digital sharing of each user's planned flight details. 
Each user will have the same situational awareness of the airspace, unlike what happens in today's air traffic control \cite{nasa_utm_2021}.

\Gls{UAM} vehicles typically operate at lower altitudes, often within urban areas, requiring a different set of rules and procedures for separation, navigation, and \gls{ASM}, compared to traditional aviation, which predominantly operates at higher altitudes.
\Gls{UAM} vehicles expected to operate with much higher traffic density than conventional aviation. This necessitates advanced automation and communication systems for collision avoidance, route planning, and real-time traffic management.
\Gls{UAM}  designed for shorter point-to-point trips, often serving as urban air taxis. This requires a different approach to traffic management, including take-off and landing procedures, queuing at landing pads, and managing very short flight segments.
Many \gls{UAM} are electric or hybrid-electric, leading to differences in energy management and charging infrastructure requirements. UTM for AAM must consider these factors when planning routes and ensuring availability of charging infrastructure.
\Gls{UAM} may involve decentralized takeoff and landing locations, such as vertiports, rooftops, and helipads. UTM systems must manage these multiple points of origin and destination, requiring coordination and communication between various stakeholders.
\cite{carc_amc}.


\subsection{Challenges of Integration}

Integrating helicopters, or in general vertical take-off landing vehicles (VTOL), into the air traffic flow is a challenge due to their special performance characteristics compared to fixed0wing aircraft, resulting in non-optimal usage of airport capacity \cite{Schuchardt_2023}.


\subsubsection{Maturation, Validation, and Deployment of U-space Services (U1 and U2)}
\subsubsection{Development of Advanced U-space Services (U3 and U4) for High Traffic and Complex Scenarios}
\subsubsection{Enabling UAM through Autonomous Operations Integration in Complex Airspace}
\subsubsection{Integration of ATM, UAM, and U-space Systems and Interfaces}
\subsubsection{Addressing Social Acceptance, Environmental Impacts, and Sustainability in UAM}
\subsubsection{U-space Application Concepts for Very Low Level (VLL) Airspace}




