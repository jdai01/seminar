\section{Background}

% How current ATM works:
%     Human controllers, radar systems, communication systems.
% Limitations of current systems:
%     Limited scalability
%     High workload
%     Human error
% Early applications of AI in aviation (e.g., scheduling, routing, predictive maintenance).

In the early days of \gls{ATM}, \gls{ATC} was a largely manual process which heavily relied on visual observations and human judgement. 
As the industry expanded, the need for more advanced, reliable, and efficient technologies emerged to manage the exponential growing air traffic.
Modern \gls{ATM} system are built around a network of surveillance (e.g. radar, \gls{ADS-B}),  communication (e.g. radio), and navigation (e.g. satelite and \gls{GPS}) technologies, all coordinated by human operators \cite{careerroo2024technology}.  
Air traffic controllers issue clearances, separate aircraft to maintain safe distances, and manage takeoffs and landings based on real-time information and experience \cite{dfs2025controller}.

However, this model has several shortfalls. 
Human controllers are subject to workload constraints and cognitive overload, especially in congested airspace \cite{Meier_2024}.
As traffic increases, the system becomes harder to scale. 
Moreover, human reaction times and decision-making are limited compared to what could be achieved with intelligent, automated systems.

With the current shortage of \gls{ATC} operators and the expected growth of air traffic, the need for \gls{AI} tools in \gls{ATM} is of increasing demand \cite{Meier_2024}.
Though automations already exist in some areas \cite{skybrary2025automation}, these systems often follow fixed rules and lack adaptability \cite{Meier_2024}. 
Integrating autonomous aircrafts and \gls{UAM} operations into this system would not only stress current infrastructure but also require faster, more adaptive decision-making \cite{Rumba_2020} -- precisely where AI offers value .

