\section{Current Research Areas of AI in ATM}

The rapid growth of global aviation presents significant challenges for ATC operations, with such being and not limited to airspace congestion and traffic complexity, human cognitive limitations in \gls{ATC} operations, environmental and weather uncertainties, increasing demand for fuel efficiency and sustainability, and cybersecurity and safety risks. 
Together with \gls{UAM} and \glspl{UAV}, the have further complicated \gls{ASM}.
These challenges necessitate the integration of \gls{AI} to assist human decision-making, improve overall efficiency, and ensure safety in these increasing complex air traffic environments \cite{Ramachandran_2025}. 

The following subsections describe the current research areas of \gls{AI} in \gls{ATM}  \cite{Ramachandran_2025}.


\subsection{Real-Time Decision Support and Conflict Resolution}

Decision support systems utalise advanced \gls{ML} algorithms and neural networks to deliver real-time risk assessments and support dynamic decision making processing. 
Its ability to automate repetitive tasks and process real-time data assist \glspl{ATCO} by offering precise risk evaluations and actionable recommendations, improving situational awareness \cite{Meier_2024}. 

In conflict detection and resolution, \gls{AI} systems not only detect potential conflicts, but also generate appropriate resolutions that align with \glspl{ATCO}' typical strategies \cite{Meier_2024}.
Through reinforcement learning and multi-agent coordination, faster and more precise resolutions can be generated \cite{Ramachandran_2025}.

% These systems aim to increase the acceptance and trust in AI-generated solutions by aligning them with the decision-making preferences of individual ATCOs.

% An example is the \gls{iCS}, which involves \glspl{ATCO} to generate conflict scenerios and learning from their resolution strategies \cite{Meier_2024}.



\subsection{Predictive Analytics for Traffic Flow Optimisation}

This is a research area where \gls{AI} is explored to anticipate and manage air traffic demand, congestion, and delays more efficiently.
\gls{AI} algorithms can forecast traffic congestion patterns using historical and real-time flight data \cite{Ramachandran_2025}. 
Trajectory and path planning utalises \gls{AI} to enhance 4D trajectory planning by enabling real-time dynamic re-planning during flight, responding to weather, or airspace restrictions without manual interventions \cite{Meier_2024}. 
To tackle the \gls{ASSP}, techniques like \gls{TBS} and \gls{PSO} are used to optimise arrival order and timing, reducing delays at busy airports \cite{Meier_2024}. 
Developments in prediction systems relating to weather and environmental factors are done by combining radar and weather data, aiding in route planning and reducing weather-related delays \cite{Meier_2024} \cite{reynolds2023weather}. 
The use of \glspl{GNN} and Transformer models optimise airspace sectorisation dynamically. 



\subsection{Intelligence Communication Between Pilots and Controllers}

\gls{AI}-driven speech recognition systems (e.g. \gls{LLM}-powered) transcribe and interpret \gls{ATC} communications. 
With \gls{ASR}, it generates information table by processing voice communication transcripts, which serves as references for producing potential taxi plans and calculating the surface movement collision risk \cite{Pang_2025}.

Such models like OpenAI o3 and Gemini 2.0 reduce misunderstandings and improve controller-pilot interactions \cite{Ramachandran_2025}. 



\subsection{AI-Driven Automation for UAS}

\Gls{aUTM} is enabled with \gls{AI}, preventing conflicts between manned and unmanned aircrafts.
% \Gls{MARL} enhances swarm intelligence for \gls{UAS} coordination \cite{Ramachandran_2025}.

This is a rapidly growing research field, especially for integrating drones safely into controlled airspaces.


\subsection{Enhancing Safety through AI-Powered Surveillance and Monitoring}

\gls{AI} is being researched to detect anomalies, monitor aircraft behaviour, and improve situational awareness using sensor fusion and video analytics.

\gls{AI}-based radar and satellite tracking improve aircraft detection and monitoring.
Computer Vision models analyse runway occupancy and ground movement for airport safety \cite{Ramachandran_2025}.