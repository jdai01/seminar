\section{Foundation Enablers}






\subsection{Satallite-Based Navigation \& Communication}

The satalite-based system uses \gls{GPS} and \gls{GNSS} to improve accuracy of flight paths. 
Compared to the traditional ground-based \gls{ATM}, the satalite-based system offers real-timereal-time and global coverage of air traffic across the globe. 
Their integration to \gls{ATM} reduces error caused by outdated or incomplete radar data, which can lead to miscommunication or unsafe distance separation between aircrafts.

The transition towards a satalite \gls{ATM} offers a safer and more efficient \gls{ATM} as it helps \glspl{ATCO} to make better informed decisions even in remote or congested airspaces \cite{atmexcite2025satellite}.



\subsection{Infrastructure Upgrades}

The legacy (or outdated) systems are not designed to support the precision and real-time data provided by satellite technologies, it calls infrastructure upgrades \cite{atmexcite2025satellite}. 

For example, in \gls{US}, under \gls{NextGen}, from the \gls{FAA} has revamped \gls{ATC} infrastructure for communications, navigation, surveillance, automantion, and information management to increase the safety, efficiency, capacity, predictability, flexibility, and resiliency of \gls{US} avaiation \cite{faa_nextgen}. 
This aims to revitalise \gls{FAA} infrastructure with new radar systems, air traffic control terminals, and increased hiring. 
This includes the replacement ``outdated technologies'' such such as binoculars for visual checks and floppy disks for data storage.
There is also a need to transition from ageing copper wire telecommunications to modern alternatives such as fibre, wireless, and satellite networks \cite{airporttech_us_upgrade}.





\subsection{Remote \& Digital Towers}

\gls{RTS} is a system which allows aerodrome \gls{ATC} or \gls{FIS} to be provided from a location other than the aerodrome (away from the physical airport tower) whilst maintaining a level of operational safety which is equivalent to that achievle using a manned Tower at the aerodrome to oversee both air and ground movements \cite{skybrary_rts}. 
An \gls{RTS} combines advanced technology, real-time visual feeds and efficient communication to enhance air traffic control while allowing controllers to operate remotely.
An \gls{RTS} requires a number of high-definition cameras/sensors along with a vast network of signal cabling equipment to allow for fast data transfer (without lag) ensuring seamless communication between the controller and the aerodrome.

Integrating \gls{RTS} has helped with cost savings, enhanced technology, staffing issues, and safety improvements \cite{globalaero2024remote}. 
