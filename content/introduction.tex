\section{Introduction}

% Define Air Traffic Management (ATM).
% Briefly introduce Urban Air Mobility (UAM) and autonomous airliners.
% Present the core problem: increased airspace complexity and traffic.
% Introduce AI as a solution.
% State the objective of the report.

\subsection{Artificial Intelligence}

\Gls{AI}, along with its subfields \gls{ML} and \gls{DL}, has emerged as fundamental tools in the modernisation and optimisation of various industries, including aviation \cite{Tafur_2025}.
\gls{ML} involves the development of algorithms and models that enable computers to learn from data and make informed decisions without being explicitly programmed \cite{Tafur_2025}. 
\gls{DL}, a subset of \gls{ML}, utilises artificial neural networks that mimic the structure and function of the human brain to process complex patterns and large datasets \cite{Tafur_2025}. 
In combination with \gls{AI}-driven tools such as \gls{NLP} and computer vision, , these tools are being integrated across a wide range of aviation applications, from flight operations and maintenance to passenger services and air traffic management \cite{Tafur_2025}.


\subsection{Urban Air Mobility}
According to \gls{EASA}, \gls{UAM} refers to an air transportation system for passengers and cargo in urban environments, with the transportation performed by an electric aircraft capable of \gls{VTOL}, either remotely piloted or with a pilot on board. 
Commercial operations in European cities are expected to begin as early as 2025, with initial applications focusing on drone-based goods delivery and passenger transport using piloted aircraft \cite{easa_uam}. 
Several pilot projects are currently underway. 
European manufacturers such as Airbus, with its CityAirbus NextGen, and Volocopter, with its VoloDrone, are actively developing and testing aircraft designed for both passenger and cargo transport \cite{gomez2024uam}.

The anticipated benefits of \gls{UAM} include faster and more sustainable transportation, reduced congestion, and extended urban connectivity. 
However, major concerns remain regarding safety, environmental impact, noise pollution, and cybersecurity.
Public acceptance and user confidence will be critical factors for the successful implementation of \gls{UAM} in Europe \cite{easa_uamlandscape}.

Currently, many \gls{UAM} vehicles are in the development or testing phase, with a long-term objective of achieving fully autonomous operations using \glspl{UAV} \cite{Vernol_2023}.


\subsection{Autonomous Airlines}

Autonomous airliners represent a branch of \glspl{UAV}, consisting of fixed-wing aircraft capable of flying and navigating without direct intervention of a human pilot.
Although modern commercial airliners already automate approximately 93\% of flight functions, such as autopilot systems and \gls{ADS-B}, there remains a growing demand to implement higher levels of autonomy.
Increased automation is seen as a path toward enhanced safety, greater scalability, and improved affordability.

Human error is cited as the leading cause in approximately 80\% of general aviation accidents. 
As a result, the vision for autonomous airliners includes minimising single points of failure in both design and operation by removing the human from direct aviate-navigate-communicate roles and replacing them with reliable, intelligent automation \cite{wisk2022autonomous}. 

While fully autonomous aircraft are technically feasible today, they have not yet been deployed for public use \cite{wisk2022autonomous}.
This limited adoption can be attributed to several key factors \cite{Vance_2019}:
\begin{enumerate}[label=\alph*)]
    \item public acceptance rates remain below the 50\% threshold typical for early adopters of innovative technologies,
    \item persistent public trust in the value of direct human pilot presence and intervention, and
    \item unresolved regulatory and cybersecurity concerns.
\end{enumerate}


\subsection{Air Traffic Management}

\Gls{ATM} refers to the systems and services that ensure the safe anf efficient movement of aircrafts during all phases of operations, through controlled airspaces and on the ground at airports \cite{skybraryATM}. 
It comprises serveral components, including \gls{ATC} (its main component), \gls{ASM}, \gls{ATFM}, and \gls{ATS} \cite{Schuchardt_2023}.

\Glspl{ATCO} are responsible for directing aircraft safely and efficiently, managing takeoffs and landings, maintaining safe distances between aircraft en route and handling emergencies. 
Their role demands high levels of situational awareness, rapid decision-making, and the ability to manage multiple tasks under high stress conditions.
These indispensable skills, such as judgement, flexibility and the ability to handle unexpected situations, remains critical and are not easily replicated by automated systems, despite advancements in digitalisation \cite{eurocontrol2024digitalisation}.

However, the rapid expansion of commercial aviation, \gls{UAM}, and \glspl{UAV} have significantly increased the complexity of \gls{ASM} \cite{Ramachandran_2025}.
With air traffic volume increases, the scalability of the system is limited by the finite capacity of \glspl{ATCO}, who are subject to workload constraints and cognitive overload \cite{Meier_2024}.
Fatigue and informaion overload  have become key contributors to operational inefficiencies and potential safety risks \cite{Ramachandran_2025}. 
Furthermore, human limitations in reaction time and decision-making speed highlight the need for intelligent, automated support systems that can enhance overall system performance.

Integrating \gls{UAM} into air traffic flow poses a significant challenge due to their unique performance characteristics, which differ from those of fixed-wing aircraft.
These differences can lead to suboptimal use of airport capacity \cite{Schuchardt_2023}.
Compounding the issue is the current shortage of \glspl{ATCO}, alongside the long training periods required to qualify new personnel, this has amplified the demand for \gls{AI}-based solutions in \gls{ATM}. 
\gls{AI} technologies are able to solve these challenges through real-time data processing, predictive analytics, and autonomous decision-making capabilities \cite{Ramachandran_2025}. 
While some automations already exist in some areas \cite{skybrary2025automation}, existing systems often rely on rigid rule-based frameworks and lack the flexibility and adaptability needed for dynamic environments \cite{Meier_2024}. 

The integration of \gls{UAV} into \gls{ATM} has led to the development of a new branch known as \gls{UTM}.
As \glspl{UAV} often operate across both controlled and uncontrolled airspaces -- and \glspl{ATCO} are only responsible for controlled airspaces -- a key challenge arises: there is no \gls{ATC} service in uncontrolled airspaces.
This lack of oversight increases the risk of mid-air collisions or accidents involving other \glspl{UAV}, manned aircraft, ground vehicles, or natural and artificial obstacles.
Therefore, a dedicated system like \gls{UTM} is essential to ensure safe and efficient \gls{UAV} operations in all types of airspace \cite{Zsolt_2017}.

The integration of \gls{UAM} and \gls{UTM} into the existing \gls{ATM} framework will not only stress current infrastructure but also require faster, more adaptive decision-making \cite{Rumba_2020} -- an area where \gls{AI} technologies can provide substantial value.

% This report explores the roles of \gls{AI} in shaping the future of \gls{ATM}, focusing particularly on its application to autonomous airliners and \gls{UAM} integration.