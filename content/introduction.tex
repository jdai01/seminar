\section{Introduction}

% Define Air Traffic Management (ATM).
% Briefly introduce Urban Air Mobility (UAM) and autonomous airliners.
% Present the core problem: increased airspace complexity and traffic.
% Introduce AI as a solution.
% State the objective of the report.

\Gls{ATM} refers to the systems and services that ensure the safe anf efficient movement of aircrafts during all phases of operations, through controlled airspaces and on the ground \cite{skybraryATM}. 
Traditionally, this has been a highly human-centered system, relying on air traffic controllers, pilots, and pre-defined procedures.
However, as global air traffic continues to rise and with emerging aerial technologies such as \gls{UAM} and autonomous airliners, the complexity of managing airspace is set to increase dramatically \cite{Schuchardt_2023}.

\gls{UAM}, which includes \gls{VTOL} aircrafts, such as helicopters, operating in densely populated urban areas, introduces a new dimension of aerial activity.
These vehicles are expected to operated at lower altitudes and with higher frequencies compared to traditional aircraft, leading to increased airspace density, especially near cities \cite{Schuchardt_2023}.
Simultaneously, advances in autonomous flight systems are enabling a shift towards single pilot operated aircraft or even pilotless airliners in the future \cite{Vance_2019}.

\gls{AI}, machine learning and deep learning, branches of \gls{AI}, have emerged as fundamental tools in addressing the challenges of this evolving landscape.
From predictive analytics to real-time decision-making and autonomous coordination, \gls{AI} has the potential to transform how we manage air traffic.
The increase in air traffic density and increasing volume of information sending through the system, it is necessitating more efficient optimization algorithms to maintain safety and efficiency in the airspace \cite{Tafur_2025}.

This report explores the roles of \gls{AI} in shaping the future of \gls{ATM}, focusing particularly on its application to autonomous airliners and \gls{UAM} integration.