\section{Remote and Digital Tower Service}

\subsection{Introduction}

\gls{RTS} is a system which allows aerodrome \gls{ATC} or \gls{FIS} to be provided from a location other than the aerodrome (away from the physical airport tower) whilst maintaining a level of operational safety which is equivalent to that achievle using a manned Tower at the aerodrome to oversee both air and ground movements \cite{skybrary_rts}. 
An RTS combines advanced technology, real-time visual feeds and efficient communication to enhance air traffic control while allowing controllers to operate remotely.
An RTS requires a number of high-definition cameras/sensors along with a vast network of signal cabling equipment to allow for fast data transfer (without lag) ensuring seamless communication between the controller and the aerodrome.


Its first opperational approval in the world was granted to Swedisch ANSP Luftfartsverket by the Swedisch Transport Agency in October 2014.
Since then, only several countries and regions have embrached \gls{RTS} \cite{globalaero2024remote}, 
As of writing of this report, \gls{RTS} is currently not approved for use in \gls{NAS} in the US \cite{faa_remote_tower}.


\subsection{Example: London City Airport \cite{lcy2022digital}}
An example would be the \gls{ATC} for London City Airport (LCY) is performed in NATS's air traffic control centre in Swanwick, Hamsphire, approximately 108 miles (173 km) away from LCY, with operations commenced in 2021. 
At LCY, \gls{ATC} is provided by a digital system, which sees 14 high-definition cameras and sensors mounted ona landside mast at the airport, which provide a 360 \textdegree view of the airfield. 




\subsection{Advantages}
\gls{RTS} is employed for the following reasons:
\begin{itemize}
    \item Cost savings: Building a remote tower typically requires a smaller unoccupied structure to house the cameras and equipment, whereas a conventional tower requires substantially more space. This means the construction costs are far lower for a remote ATC. Additionally, the maintenance costs on a conventional tower are higher due to the physical windows, radar screens and other specialised equipment.
    \item Enhanced technology. Due to advancements in technology, air traffic controllers can observe aircraft through poor visibility or at night using infrared or high-definition cameras, or they can also use the enhanced zoom function to potentially spot wildlife on the runway that might not be visible with the naked eye. The remote ATC has advanced even further, providing an array of screens with detailed data such as runway conditions and other critical information. This data is fed into the livestream, meaning the controller doesn’t need to switch views constantly.
    \item Staffing. Remote ATC towers are based at one location, which means controllers are no longer present at each airport. For aerodromes that have infrequent usage, you could use one controller to manage multiple aerodromes. This consolidation minimizes staff requirements, which results in cost savings. Given that controllers no longer need to work in a remote environment, there is a better work-life balance, which results in better workforce morale.
    \item Safety Improvements. High-definition cameras and infrared technology give controllers better visibility, especially in nighttime operations. This assists with the monitoring of potential hazards. Additionally, through the integration of tracking technology, various sensors can offer multiple viewing angles, improving situational awareness.
\end{itemize}



\subsection{Challenges}

Inplementing an \gls{RTS} requires addressing specific challenges, including:

\paragraph{Adapting to new workflows}
New workflows/systems will be implemented to \glspl{ATCO} at remote towers. 
Even with the new technology, \glspl{ATCO} must maintain the same level of dilligence like one would be at a conventional control towers: addressing potential distractions and fatigue.

If \glspl{ATCO} are expected to manage multiple airports, this would be more challenging. 


\paragraph{Security and resilience}
As remote towers rely heavily on technology and uninterrupted data transfer, they must be protected against cyber threats and extreme weather conditions.


\paragraph{Human factors}
Remote \glspl{ATC} can alleviate staff shortages as they require fewer controllers to operate. 
However, individuals may face stress due to increased workload because they will need to adapt to the new technology and new processes.

\paragraph{Operation efficiency}
Remote ATC technology is sensitive and operators will need to ensure that the equipment is properly maintained. 
Maintenance will be particularly challenging in areas of extreme weather such as low temperatures or high wind. 
If an RTS is based in a remote location, they may be lacking access to a pool of skilled operators.

