\section{Role of AI in Future ATM}

Given the increasing complexity of \gls{ASM}, \gls{AI} is now a fundamental enabler of next-generation \gls{ATC} systems. 
\gls{AI}-driven automation and decision-support tools help metigate risks, reduce operational inefficiencies, and enhance safety \cite{Ramachandran_2025}.
Main research focuses are on traffic, trajectory, performances, airports, weather, accident, fuel, passenger, and time \cite{Tafur_2025}.
The following are reasons why \gls{AI} is indispensable for modern \gls{ATC}.


\subsection{Real-Time Decision Support and Conflict Resolution}

\gls{AI}-powered decision-support systems process vast amounts of real-time data to assist \glspl{ATCO}, enabling faster and more precise conflict detection and resolution that align with \gls{ATCO}'s typical strategies \cite{Ramachandran_2025}.
Its key advantage over traditional approaches lies in their ability to automate repetitive tasks that typically consume time and cognetive resource, reducing the workload of \glspl{ATCO} , allowing them to focus more effectivly on complex, unpredictable challenges that demand higher-level decision-making skills \cite{Meier_2024}. 
An example is the \gls{iCS}, which involves \glspl{ATCO} to generate conflict scenerios and learning from their resolution strategies \cite{Meier_2024}.



\subsection{Predictive Analytics for Traffic Flow Optimisation}
\gls{AI} algorithms can forecast traffic congestion patterns using historical and real-time flight data, with Graph Neural Networks and Transformer models optimising airspace sectorisation dynamically \cite{Ramachandran_2025}. 
Two main research areas have been identified: trajectory and path planning, and separation and sequencing.

\subsubsection{Trajectory and Path Planning}
Trajectory and path planning allows aircraft to autonomously adjust paths in real time, improving scalability compared to traditional centralised methods \cite{Tafur_2025}.
\gls{AI} enhances 4D trajectory planning by enabling real-time dynamic re-planning during flight, responding to traffic, weather, or air space restructions without manual interventions. 
These \gls{AI} systems continously optimise routes based on live data, balancing fuel efficiency, flight time, and safety \cite{Meier_2024}.

\subsubsection{Separation and Sequencing}
Tools are developed to tackle the \gls{ASSP}, with techniques such as \gls{TBS} and \gls{PSO} to optimise arrival order and timing, reducing delays at busy airports.
\gls{TGP} model helps maintain safe distances between aircrafts, even under uncertainty. while offering insights to flight dynamics during the final approach \cite{Meier_2024}. 

\subsubsection{Meteological and environmental factors}
Severe weather conditions such as storms, turbulence, and wind shear create safety risks for aircraft. 
Traditional weather forecasting models struggle to provide precise, real-time impact predictions for \gls{ATC} decision making \cite{Ramachandran_2025}.
\gls{AI} prediction systems are developed to combine radar and weather data, aiding in route planning and reducing weather-related delays. 
Advanced visualisation techniques, such as five-dimensional displays, provide \glspl{ATCO} with real-time views of weather conditions and trajectory data, improving situational awareness \cite{Meier_2024}. 


\subsection{Intelligence Communication Between Pilots and Controllers}
\gls{AI}-driven speech recognition systems (e.g. \gls{LLM}-powered) transcribe and interpret \gls{ATC} communications. 
Such models like OpenAI o3 and Gemini 2.0 reduce misunderstandings and improve controller-pilot interactions \cite{Ramachandran_2025}. 


\subsection{AI-Driven Automation for UAS}
\gls{AI} enables autonomous \gls{UAV} traffic management, preventing conflicts between manned and unmanned aircraft.
\gls{MARL} enhances swarm intelligence for \gls{UAS} coordination.


\subsection{Enhancing Safety through AI-Powered Surveillance and Monitoring}
\gls{AI}-based radar and satellite tracking improve aircraft detection and monitoring.
Computer Vision models analyse runway occupancy and ground movement for airport safety.