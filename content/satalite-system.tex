\section{Transition to Satalite Systems}

Ground-based \gls{ATM} is the traditional system that uses infrastructure on the ground that supports the safe and efficient movement of aircrafts \cite{skybraryATM}.
Examples include radar systems, flight control centers with \glspl{ATCO} and radio communications \cite{atmexcite2025satellite}.
This system is often limited by geography and often struggles with tracking planes in areas outside its range \cite{atmexcite2025satellite}.

The move towards a satalite \gls{ATC} brought about significant advancements in \gls{ATM} and offer real-time, global coverage of air traffic across the world's skies \cite{atmexcite2025satellite}.
This advancement is particularly crucial as \gls{ATC} is no longer confied to specific regions like in the traditional system and offer a safer and more efficient \gls{ATM} in even the most congested or isolated airspace \cite{atmexcite2025satellite}.

\gls{GPS} and \gls{GNSS} are examples of satallite-based systems that can significantly improve the accuracy of flight paths.
Their integration to \gls{ATM} reduces errors caused by outdated or incomplete radar data, which can lead to miscommunication or unsafe distance separation between aircrafts \cite{atmexcite2025satellite}. 
Aircrafts equipped with \gls{GPS} and \gls{GNSS} can safely navigate in regions previously deemed difficult to manage, such as remote or congested airspaces \cite{atmexcite2025satellite}. 
As these systems provide accurate and real-time position information, these satellite systems help \glspl{ATCO} to make better informed decisions, ultimately reducing the likelihood of collisions \cite{atmexcite2025satellite}. 

Legacy systems were not designed to support the precision and real-time data provided by satellite technologies, and calls for a need for significant infrastructure upgrades \cite{atmexcite2025satellite}. 
This will be further explained in the next section.