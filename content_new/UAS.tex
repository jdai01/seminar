\section{Unmanned Aerial Systems}

\subsection{Introduction}

\Gls{UAS} is a system with aircrafts that have no direct intervention from human pilot.
It consists of three components: (1) an \gls{UAV}, (2) a \gls{RPS} or \gls{GCS}, and (3) a \gls{C2} system.
An \gls{UAV} is an aircraft that operates or is designed to operate autonomously, or to be pilot remotely without a pilot onboard.
Its use cases are and not limited to security surveillance, emergency response, and small package and bulk cargo transport \cite{skybrary_uas}. 


\subsection{Urban Air Mobility}

According to \gls{EASA}, \gls{UAM} refers to an air transportation system for passengers and cargo in urban environments.%, with the transportation performed by an electric aircraft capable of \gls{VTOL}, either remotely piloted or with a pilot on board \cite{easa_uam}. 
It primarily consists of \gls{eVTOL} aircrafts, with air taxis used for passengers and drones for delivery of cargos, surveillance, and photography. 
These aircrafts can either be remotely piloted or with a pilot on board \cite{easa_uam}. 
If these aircrafts are capable of autonomous flying, it is considered a part of \gls{UAV}.

It differs from traditional aircrafts with its use of rotors, and the ability to take off and land vertically from almost anywhere with a suitable platform (vertiports) whereas traditional aircrafts are mostly equipped with fixed wings and require runways to operate.
Its range of opperation also differs, with \gls{UAM} operate in urban areas (and possibly into remote areas) while traditional aircrafts are able to operate for long distance travel, but only to locations with runway availability.
% Example of \gls{UAM} aircrafts include \gls{eVTOL} aircraft like the VoloCity (developed by Volocopter) and CityAirbus NextGen (developed by Airbus) \cite{gomez2024uam}.
% Many European manufacturers are actively developing and testing aircrafts designed for both passenger and cargo transport \cite{gomez2024uam}.
% Starting from 2025, Commercial operations in European cities are expected, with initial applications focusing on drone-based goods delivery and passenger transport using piloted aircraft  \cite{easa_uam}. 

% The anticipated benefits of \gls{UAM} include faster and more sustainable transportation, reduced congestion, and extended urban connectivity. 
% However, major concerns remain regarding safety, environmental impact, noise pollution, and cybersecurity.
% Public acceptance and user confidence will be critical factors for the successful implementation of \gls{UAM} in Europe \cite{easa_uamlandscape}.

% Currently, many \gls{UAM} vehicles are in the development or testing phase, with a long-term objective of achieving fully autonomous operations using \glspl{UAV} \cite{Vernol_2023}.

\subsubsection{Benefits of UAM}

\Gls{UAM} is envisioned as an alternative for inter-city or intra-city travelling, providing a faster and efficient alternative to land travel. 
This would help to reduce the amount of vehicles on roads, elleviating traffic congestion and moving some of it to the sky. 
With \glspl{eVTOL}, extensive infrastructure does not need to be in place to be adopted compared to electric public transportation system on the ground. 
Only transportation hubs need to be built, without the need to build extended infrastructure on roads and tracks to support it. 
\cite{arbin2021uam}

\subsubsection{Challenges faced for UAM}

The safety of \glspl{eVTOL} technology needs to be proven to the public, especially with companies developing these aerial vehicles to be automated and work without the presence of a pilot.
Despite air travel having a much lower accident rate compared to road travel, there are also more risks involved. 
Air traffic is also much more heavily regulated than road traffic, and that means that policies and regulations need to be ironed out and tested before this type of transportation can be safely opened to public.
Routes that will not disrupt airport traffic, for example, would need to be worked out.
Air taxis are unable to take people directly point-to-point, rather from one station to another.
This means that integration between different modes of transport would need to be implemented in order to create an efficient and seamless travel experience.
\cite{arbin2021uam}



\subsection{Autonomous Airliners}

Autonomous airliners represent a branch of \glspl{UAV}, consisting of fixed-wing aircraft capable of flying and navigating without direct intervention of a human pilot, for passenger service.
Although modern commercial airliners already automate approximately 93\% of flight functions, such as autopilot systems and \gls{ADS-B}, there remains a growing demand to implement higher levels of autonomy.
Increased automation is seen as a path toward enhanced safety, greater scalability, and improved affordability.

% Human error is cited as the leading cause in approximately 80\% of general aviation accidents. 
% As a result, the vision for autonomous airliners includes minimising single points of failure in both design and operation by removing the human from direct aviate-navigate-communicate roles and replacing them with reliable, intelligent automation \cite{wisk2022autonomous}. 

% While fully autonomous aircraft are technically feasible today, they have not yet been deployed for public use \cite{wisk2022autonomous}.
% This limited adoption can be attributed to several key factors \cite{Vance_2019}:
% \begin{enumerate}[label=\alph*)]
%     \item public acceptance rates remain below the 50\% threshold typical for early adopters of innovative technologies,
%     \item persistent public trust in the value of direct human pilot presence and intervention, and
%     \item unresolved regulatory and cybersecurity concerns.
% \end{enumerate}

