\subsection{Maintain optimal airfraft separation in airspace}

In a research by Liu \cite{LIU2021100058}, a routing optimisation model was proposed to be used independently by multiple routing agents that allows real-time dynamic allocation of airspace volumes withr obust mechanisms for conflict resolution.
Safety standards held equal, a traffic management system built on this model can support more concurrent flights in a given volume of airspace than what existing systems are capable of, thus allowing more goods to be transported via automated aerial delivery.

In this system, each agent represents a USS executing central control over its own fleet of UAVs.
Multiple agents passively interact with each other via observing and reacting to the motion of each other's fleets.
No complicated inter-agent communication is required to achieve harmonious traffic outcomes. 
UAVs within the same fleet are cooperative, while UAVs across different fleets are semi-cooporative -- they share the same goal of maintain safe separation, but do not communicate flight intents or negotiate right of ways as do UAVs manage by the same USS.

While centralised traffic management is most conducive to efficient point-to-point air trips, the practical reality demands a more flexible architecture in which multiple private fleets can operate independently, yet harmoniously, in shared airspaces.
Sufficient separation is the most fundamental operating requirement shared among all airspace users, but how to maintain the desired level of separation in a decentralised operation environment is a core challenge to be addressed.







