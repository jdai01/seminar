\subsection{Safe Flight Pathing Through Hazardous Sky}

Natural events such as thunderstorms and volcanic ash clounds present safety challenges for the aviation section.
In addition, aerosols and gases arising from natural hazards such as forest fires and desert dust can also severely reduce visibility and damage engines.
Even space weather, such as the solar wind, can impact aviation by disrupting satallite communication and increasing radiation exposure.

The \gls{ALARM} project was funded within the framework of the \gls{SESAR} Joint Undertaking to move the aviation sector towards mordenising Europeans gls{ATM} system by developing a prototype monitoring and early warning system for various hazards.
To achieve this, near real-time data from ground-based and satellite systems was gathered. This highly granular information was then processed and fed into models for identifying the displacement of particles and gases derived from natural hazards, as well as extreme weather situations. 
The first step is to provide a snapshot of what is happening, then to develop predictive models that is able to provide the aviation sector with forecasts of between one hour ahead and a day ahead. 
To achieve this, \gls{AI} was applied to observational data and historical observation and create a model that is able to learn from past localised forecasts and weather observations, in order to be able to better predict the likely evolution of any given natural event.
This model can be used, for example, to accurately predict the behaviour of severe thunderstorm over an airport, and \gls{ATCO} can use this information to make in-flight deviations or reschedule flights altogether.

\cite{cordis2022aiatm}
