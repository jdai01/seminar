\section{Urban Air Mobility}

According to \gls{EASA}, \gls{UAM} refers to an air transportation system for passengers and cargo in urban environments, with the transportation performed by an electric aircraft capable of \gls{VTOL}, either remotely piloted or with a pilot on board. 
Commercial operations in European cities are expected to begin as early as 2025, with initial applications focusing on drone-based goods delivery and passenger transport using piloted aircraft \cite{easa_uam}. 
Several pilot projects are currently underway. 
European manufacturers such as Airbus, with its CityAirbus NextGen, and Volocopter, with its VoloDrone, are actively developing and testing aircraft designed for both passenger and cargo transport \cite{gomez2024uam}.

The anticipated benefits of \gls{UAM} include faster and more sustainable transportation, reduced congestion, and extended urban connectivity. 
However, major concerns remain regarding safety, environmental impact, noise pollution, and cybersecurity.
Public acceptance and user confidence will be critical factors for the successful implementation of \gls{UAM} in Europe \cite{easa_uamlandscape}.

Currently, many \gls{UAM} vehicles are in the development or testing phase, with a long-term objective of achieving fully autonomous operations using \glspl{UAV} \cite{Vernol_2023}.


\


\subsection{Integration Challenges}