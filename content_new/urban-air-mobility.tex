\section{Urban Air Mobility}

\subsection{Introduction}

According to \gls{EASA}, \gls{UAM} refers to an air transportation system for passengers and cargo in urban environments, with the transportation performed by an electric aircraft capable of \gls{VTOL}, either remotely piloted or with a pilot on board \cite{easa_uam}. 
Example of \gls{UAM} aircrafts include \gls{eVTOL} aircraft like the VoloCity (developed by Volocopter) and CityAirbus NextGen (developed by Airbus) \cite{gomez2024uam}.
Many European manufacturers are actively developing and testing aircrafts designed for both passenger and cargo transport \cite{gomez2024uam}.
Starting from 2025, Commercial operations in European cities are expected, with initial applications focusing on drone-based goods delivery and passenger transport using piloted aircraft  \cite{easa_uam}. 

The anticipated benefits of \gls{UAM} include faster and more sustainable transportation, reduced congestion, and extended urban connectivity. 
However, major concerns remain regarding safety, environmental impact, noise pollution, and cybersecurity.
Public acceptance and user confidence will be critical factors for the successful implementation of \gls{UAM} in Europe \cite{easa_uamlandscape}.

% Currently, many \gls{UAM} vehicles are in the development or testing phase, with a long-term objective of achieving fully autonomous operations using \glspl{UAV} \cite{Vernol_2023}.


\subsection{Unmanned Aerial Systems}

\Gls{UAS} is a branch of \gls{UAM}, with aircrafts with no direct intervention from human pilot.
It is also referred to as drones, and consists of three components: (1) an \gls{UAV}, (2) an autonomous or human-operated control system, and (3) a command and control (C2) system (sometimes referred to communication, command, and control (C3) system) \cite{skybrary_uas}.
An \gls{UAV} is an aircraft that operates or is designed to operate autonomously, or to be pilot remotely without a pilot onboard \cite{skybrary_uas}.

Currently, it is being used for security surveillance, emergency response, small package and bulk cargo transport, and furthermore \cite{skybrary_uas}. 
Its use cases is not listed exhaustively here. 



\subsection{Autonomous Airliners}

Autonomous airliners represent a branch of \glspl{UAV}, consisting of fixed-wing aircraft capable of flying and navigating without direct intervention of a human pilot.
Although modern commercial airliners already automate approximately 93\% of flight functions, such as autopilot systems and \gls{ADS-B}, there remains a growing demand to implement higher levels of autonomy.
Increased automation is seen as a path toward enhanced safety, greater scalability, and improved affordability.

% Human error is cited as the leading cause in approximately 80\% of general aviation accidents. 
% As a result, the vision for autonomous airliners includes minimising single points of failure in both design and operation by removing the human from direct aviate-navigate-communicate roles and replacing them with reliable, intelligent automation \cite{wisk2022autonomous}. 

% While fully autonomous aircraft are technically feasible today, they have not yet been deployed for public use \cite{wisk2022autonomous}.
% This limited adoption can be attributed to several key factors \cite{Vance_2019}:
% \begin{enumerate}[label=\alph*)]
%     \item public acceptance rates remain below the 50\% threshold typical for early adopters of innovative technologies,
%     \item persistent public trust in the value of direct human pilot presence and intervention, and
%     \item unresolved regulatory and cybersecurity concerns.
% \end{enumerate}

