\section{Body} %% <= Insert title

The following part describes \LaTeX commands and shows examples. Furthermore, the structure is shown in sub-items.

\subsection{Literature and references} %% <= Insert title

All citation specifications (e.g. direct and indirect quotations) as well as specifications for listing the sources in the bibliography (e.g. anthologies or monographs) can be found in the document "Citing according to Harvard" from the THI University Library.

The short reference is in square brackets [ ] in the text, here is an example\cite{Dreamer2025a}. When referring to a specific page, it looks like this\cite[S. 225]{Dreamer2025a}.



\subsection{How to equations} %% <= Titel einfügen
You can write equations in line ${a^2 + b^2 = c^2}$. Or you can write an equation as paragraph
\begin{equation}
    \label{thisisasum}
    \sum_{i=1}^K x_i = 10.
\end{equation}
Sometimes you don't like equation numbers:
\begin{equation*}
    \sum_{i=1}^K x_i = 10.
\end{equation*}
However, a number is great because you can refere to it, i.e., Equation \ref{thisisasum} shows a sum.
It is also possible to write equation systems:
\begin{align}
    a + b &= 2 \nonumber\\
    b & = 3 \nonumber\\
    a &= ? \label{whatisa}
\end{align}
The task is to solve equation \ref{whatisa}.
It's all possible but you may have to look for it.
A good start might be:
\url{https://www.overleaf.com/learn/latex/Mathematical_expressions}

\subsection{How to tables}
A table is as easy as everything else. Have a look at Table \ref{table}.
\begin{table}[H]
    \caption{this is a table}
    \label{table}
    \small
    \centering
    \setlength{\tabcolsep}{4.5pt}
    \begin{tabular}{|c|c|c|}
        \hline
         & layer context & Affine  \\
        \hline
            a  & [t-2,t+2]     & ($5 \times 40)  \times 512$  \\
            b  & \{t-2,t,t+2\} & ($3 \times 512) \times 512$  \\
            c  & \{t-2,t,t+2\} & ($3 \times 512) \times 512$  \\
            d  & \{t\}         & $512 \times 512$  \\
            e  & \{t\}         & $512 \times 512$  \\
            f  & [0,T) & N/A  \\
            g & [0,T) & 1024 $\times$ 256  \\
            h  & [0,T) & 256 $\times$ N \\
        \hline
    \end{tabular}
\end{table}

\subsection{How to images}
Images are also very easy. Have a look at Figure \ref{aimotionlogo}.
The position of the image is computed, automatically.
You can influence the image position but it's usually not required. 
\begin{figure}[t]
	\centering
	\includegraphics[width=15em]{thi_AImotion_logo.jpg}
	\caption{This is our logo.}
	\label{aimotionlogo}
\end{figure}

\subsubsection{Deep structure}

A deeper breakdown into the 3rd level is also possible. There are no plans for seminar papers that go deeper than level 3. Another structuring system in this case would be, for example, commonly used bullet points.

\subsection{How to code}
It is also easy to add code in your text.
However, there is almost no reason to put code in a text....
The perfect place for code is github (or the apendix).
\lstset{language=Python}
\lstset{frame=lines}
\lstset{caption={Insert code directly in your document}}
\lstset{label={lst:code_direct}}
\lstset{basicstyle=\footnotesize}
\begin{lstlisting}
from brg.datastructures import Mesh
     
mesh = Mesh.from_obj('faces.obj')
mesh.draw()
\end{lstlisting}

